\begin{spell}{Contingency}{PHB}{\nth{6} level evocation}
{
    \spTime{10 minutes}
    \spRange{Self}
    \spComponents{V, S, M  (a statuetle of yourself carved
    from ivory and decorated with gems worth at least 1,500~gp)}
    \spDuration{10 days}
}
Choose a spell of \nth{5} level or lower that you can cast, that
has a casting time of 1 action, and that can target you.
You cast that spell -- called the contingent spell -- as part
of casting \emph{contingency}, expending spell slots for both, but
the contingent spell doesn't come into effect. Instead, it
takes effect when a certain circumstance occurs. You
describe that circumstance when you cast the two spells.
For example, a \emph{contingency} cast with \emph{water breathing}
might stipulate that \emph{water breathing} comes into effect
when you are engulfed in water or a similar liquid.

The contingent spell takes effect immediately after the
circumstance is met for the first time, whether or not you
want it to, and then \emph{contingency} ends.

The contingent spell takes effect only on you, even
if it can normally target others. You can use only one
\emph{contingency} spell at a time. If you cast this spell again,
the effect of another \emph{contingency} spell on you ends. Also,
\emph{contingency} ends on you if its material component is
ever not on your person.
\end{spell}